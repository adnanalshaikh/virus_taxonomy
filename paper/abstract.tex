\begin{abstract}
DNA sequence classification is a critical task in the field of virology, as it plays a key role in identifying and characterizing viral pathogens. Typically, viral DNA sequences are classified by analyzing k-mers, which are substrings of length k within the DNA sequence. Recently, machine learning techniques have been utilized to improve the accuracy and efficiency of DNA sequence classification. These techniques involve training models on large datasets of known viral DNA sequences and using them to predict the classification of unknown sequences. Feature engineering techniques such as K-mer frequency encoding and Position-Specific Scoring Matrices (PSSM) have been utilized to enhance model performance. Ensemble methods like Random Forest and Gradient Boosting have also demonstrated their effectiveness in improving model performance. Incorporating these techniques holds significant potential for enhancing the accuracy and speed of viral pathogen identification and characterization. However, it is crucial to consider factors such as the choice of k-mer size, feature engineering technique, and machine learning model, as they greatly influence the model’s performance. In this study, we present our empirical investigation involving various combinations of these factors to attain the optimal performance.
\end{abstract}