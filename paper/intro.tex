\section{Introduction}

Classification, also known as taxonomy of species, is the scientific process of categorizing and organizing living organisms into a hierarchical system based on their shared characteristics. Taxonomy aims to provide a standardized and universal framework for naming and classifying species. The classification of species involves placing organisms into various ranks or categories based on their similarities and differences. The most fundamental unit of classification is the species, which represents a group of individuals that can interbreed and produce fertile offspring.
The hierarchical system of classification typically follows a structure that starts with the broadest categories and narrows down to more specific groups. The traditional hierarchy consists of eight ranks: Domain, Kingdom, Phylum, Class, Order, Family, Genus, and Species

Traditionally, taxonomy was based on morphology. It relied on observable physical attributes like size, shape, color, and body structure to classify organisms. However, with advancements in molecular biology and DNA sequencing techniques, DNA has become a valuable tool in taxonomy. DNA classification provides a more objective and accurate method for determining relationships between organisms, as DNA carries hereditary information that is passed down from generation to generation.  By using the differences in DNA among different species, scientists can organize and categorize living things based on their genetic makeup in the form of taxonomy or classification system. According to a 2020 study by Sosa et al., morphological analysis and ecological factors should still be considered to improve species recognition and classification (81). Consequently, DNA classification should not be considered the only means of identifying and classifying species. 

\begin{figure*}[t]
\begin{centering}
		\includegraphics[width=\textwidth]{Figures/acc_loss_3taxa.png}
	\caption{distribution  }
\label{fig:cnn-archit}
\end{centering}
\end{figure*}

Machine learning is an evolving field of technology that has revolutionized the way we approach and solve problems. It involves a system that can learn from data, identify patterns, and make decisions with minimal human intervention. These decisions are more efficient and quicker to process; however, the accuracy of the decisions remains dependent on the data it is exposed to \cite{gama2011}.  Machine learning is increasingly becoming a more popular and reliable method for analyzing and classifying DNA data.  It is used in a variety of ways to classify, analyse and extract useful information from DNA data sets such as automating the segmentation of DNA sequences, which enables greater accuracy and speed of sorting and analysis than traditional methods \cite{chambert2019}. This technology is also used to identify variations in the DNA bases and relate them to medical conditions \cite{Kallman2019}.
Another application is in determining how entire genomes are related to one another across different organisms \cite{cottrell2017}. Also, it is used to identify regions of DNA that are evolutionarily conserved across multiple species \cite{monsalve2019}, or to generate predictions about the function of certain genetic elements. 

Recently several ML algorithms such as Support Vector Machines and Artificial Neural Networks are used to classify features given information about the genetic data, allowing for a more automated approach \cite{kalender2020}. This is especially useful for lab professionals as it reduces the need for manual intervention, allowing for greater efficiency in the classification \cite{vihar2019}. Additionally, due to the speed and accuracy of ML algorithms, scientists can quickly identify changes in genomes and mutations, aiding in research and development of new drugs \cite{neto2021}. In conclusion, ML can be utilised in DNA classification, contributing to the automation and efficacy of the process.

Machine learning (ML) has become a powerful tool in data analysis, enabling systems to learn from data, identify patterns, and make decisions with minimal human intervention. In DNA classification, ML techniques significantly enhance efficiency and accuracy by automating processes such as DNA sequence segmentation. Algorithms like Support Vector Machines (SVMs) and Artificial Neural Networks (ANNs) are widely applied to classify genetic data, improving both the speed and precision of analysis over traditional methods. These technologies also facilitate the identification of genetic variations, the determination of genomic relationships, and the prediction of genetic element functions.

In this paper, we propose an approach for virus classification based on DNA sequences, leveraging a hierarchical classification strategy with Local Classifiers per Node (LCN). For each classifier, we first preprocess DNA sequences by replacing ambiguous nucleotides (e.g., M, N) with the most frequent nucleotide in the sequence. We then encode each nucleotide using one-hot encoding. Due to the limited number of DNA sequences for each species, we employed two data augmentation techniques to expand the training dataset. The first technique involves adding each sequence's reverse complement to the dataset, and the second splits each DNA sequence into fixed-length subsequences.

Each encoded subsequence is then passed through a one-dimensional convolutional neural network (1D CNN) to extract sequence features. These features are subsequently fed into an LSTM layer to capture sequential patterns. During prediction, we aggregate the subsequence predictions using soft voting to classify each sequence. Our approach achieved an accuracy of 95\%, demonstrating excellent performance. All our results are reproducible with code available on GitHub[2]


By combining machine learning techniques with other methods, it is possible to classify DNA sequences with higher accuracy and efficiency. This opens up numerous possibilities in the field of genetics, allowing scientists to gain deeper insight into the roles genes and proteins play in disease, development, and many other biological processes.

 


