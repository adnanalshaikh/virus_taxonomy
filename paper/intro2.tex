\section{Introduction}
Taxonomy, or species classification, is the scientific process of categorizing living organisms into a hierarchical system based on shared characteristics. Traditionally, this classification was based on morphological features such as shape, size, and body structure. However, advancements in molecular biology and DNA sequencing have made DNA analysis an essential tool in taxonomy. DNA-based classification offers a more objective and accurate method for identifying relationships between organisms, as it relies on genetic information passed down across generations. While DNA classification enhances accuracy, integrating morphological and ecological data can further improve species identification, as noted in a 2020 study by Sosa et al.

Machine learning (ML) has emerged as a transformative tool in DNA classification. ML algorithms, such as Support Vector Machines (SVMs) and Artificial Neural Networks (ANNs), automate processes like DNA sequence segmentation and genetic data classification, leading to faster, more precise analyses than traditional methods. ML enables researchers to identify genetic variations, assess genomic relationships, and predict genetic functions. Additionally, these algorithms are used to analyze DNA sequences for applications such as detecting mutations and understanding conserved genetic regions across species, which aids research in medicine and evolutionary biology.

In this paper, we propose an ML-based approach for virus classification using DNA sequences. Our method utilizes a hierarchical classification framework with Local Classifiers per Node (LCN). We preprocess DNA sequences by replacing ambiguous nucleotides (e.g., M, N) with the most common nucleotide in each sequence and then apply one-hot encoding. Due to the limited availability of DNA sequences per species, we apply data augmentation techniques: adding reverse complements of sequences and splitting each DNA sequence into fixed-length subsequences.

Each encoded subsequence is processed through a one-dimensional convolutional neural network (1D CNN) to extract sequence features, followed by an LSTM layer to capture sequential patterns. During prediction, we use soft voting to aggregate subsequence predictions and classify each sequence. Our approach achieved a 95\% accuracy, and all results are reproducible with code available on GitHub[2].

By integrating machine learning with DNA analysis, this approach allows for high accuracy and efficiency in DNA classification, opening new opportunities in genetics to better understand gene and protein roles in biological processes and disease.
